\documentclass[a4paper, 12pt]{article} % La fuente debe ser de 12pt
\usepackage[utf8]{inputenc}
\usepackage[T1]{fontenc}
\usepackage{hyperref}
\usepackage[left=3cm, right=3cm, top=3.5cm, bottom=3.5cm]{geometry} % Márgenes recomendados
\usepackage{times} % La fuente debe ser Times New Romans
\usepackage[english, spanish, es-noshorthands, es-tabla]{babel}
\usepackage[spanish]{translator}
\usepackage[style=ieee, backend=biber]{biblatex} % Bibliografía en formato IEEE
\usepackage{sectsty}
\usepackage{portada}

\sectionfont{\MakeUppercase} % Las secciones deben estar en mayúsculas
\bibliography{\jobname}

\Director{Nombre del Director}
%\Lugar{Bilbao} % Por omisión: Madrid
%\Grado{Graduado en Matemáticas e Informática} % Por omisión: Graduado en Ingeniería Informática
%\Trabajo{TRABAJO FIN DE MÁSTER} % Por omisión: TRABAJO FIN DE GRADO

\author{Nombre del Alumno}
\date{Enero de 2017}
\title{Título del trabajo}

\begin{document}
\maketitle
\null%
\newpage

\pagenumbering{roman} % La numeración debe ser romana hasta la primera sección
\tableofcontents
\newpage

\listoffigures
\listoftables
\newpage

\begin{abstract}
  \normalsize
  Aquí el texto del abstract.

  \textbf{Palabras clave:} palabra 1, palabra 2, palabra 3\ldots
\end{abstract}

\begin{otherlanguage}{english}
  \begin{abstract}
    \normalsize
    Here goes the abstract text.

    \textbf{Keywords:} keyword 1, keyword 2, keyword 3\ldots
  \end{abstract}
\end{otherlanguage}

\newpage
\pagenumbering{arabic} % Iniciamos la numeración árabe en la primera sección

\section{Introducción}

Lorem ipsum dolor sit amet, consectetur adipiscing elit. Mauris luctus malesuada
volutpat. Nam id tortor purus. Aliquam sit amet felis ipsum. Nullam ac ante
tempus, dignissim dolor a, aliquet nunc. In hac habitasse platea dictumst. In ac
lacus sit amet velit fringilla pharetra. Duis at cursus erat. Curabitur sed
pellentesque lacus. Aliquam sit amet est nec nisl cursus varius. Sed pharetra ac
neque quis imperdiet. Quisque ac tortor enim. Sed dignissim blandit orci, id
fermentum lorem commodo eget\cite{Recos}.

\section{Estado del arte}

\section{Otra sección}

\subsection{Una subsección}

\subsubsection{Otra subsección}

\section{Resultados y conclusiones}

\subsection{Pruebas}

\subsection{Conclusiones}

\printbibliography[heading=bibnumbered] % Última sección, numerada, para la bibliografía
\end{document}
